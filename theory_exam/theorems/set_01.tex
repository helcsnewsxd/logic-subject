\documentclass{article}
\usepackage[utf8]{inputenc}
\usepackage{geometry}
 \geometry{
 a4paper,
 total={170mm,257mm},
 left=20mm,
 top=10mm,
 }
\usepackage{graphicx}
\usepackage{titling}
\usepackage{lipsum}  
\usepackage{lmodern}
\usepackage{amssymb}
\usepackage{amsmath}


\title{Combo 1 de teoremas}
\author{Emanuel Nicolás Herrador}
\date{November 2024}
 
\makeatletter
\def\@maketitle{%
  \newpage
  \null
  \vskip 1em%
  \begin{center}%
  \let \footnote \thanks
    {\LARGE \@title \par}%
    \vskip 1em%
    {\large \@author\quad-\quad \@date}%
  \end{center}%
  \par
  \vskip 1em}
\makeatother

\begin{document}

\maketitle

\section*{Teorema del Filtro Primo}
Sea $(L,s,i)$ un reticulado terna distributivo y $F$ un filtro. Supongamos $x_0\in L-F$. Entonces hay un filtro primo $P$ tal que $x_0\notin P$ y $F\subseteq P$.
\subsection*{Demostración}
Consideremos $\mathcal{F}=\{F_1:F_1\text{ es un filtro},x_0\notin F_1\text{ y }F\subseteq F_1\}$. Como $F\subseteq\mathcal{F}$, entonces $\mathcal{F}\neq\emptyset$ y, por lo tanto, $(\mathcal{F},\subseteq)$ es poset.

Sea $C$ una cadena, vamos a ver que tiene cota superior:
\begin{itemize}
  \item Si $C=\emptyset$, entonces todo elemento de $\mathcal{F}$ es cota de $C$
  \item Si $C\neq\emptyset$, consideremos $G=\{x:x\in F_1\text{ para }F_1\in C\}$
        \begin{itemize}
          \item Como $C\neq\emptyset$, existe un filtro $F_1\in C$. Luego, por def. de filtro, $F_1\neq\emptyset$. Por ello, $\exists x\in F_1$ y eso significa que $x\in G$ por lo que $G\neq\emptyset$.
          \item Sean $x,y\in G$, entonces $\exists F_1,F_2\in C:(x\in F_1\land y\in F_2)$. Por def. de cadena, $F_1\subseteq F_2\lor F_1\supseteq F_2$. Sin pérdida de generalidad, supongamos $F_1\subseteq F_2$. Luego, $x,y\in F_2$ por lo que $x\ i\ y\in F_2$ por def. de filtro. Por ello, como $F_2\subseteq G$, tenemos que $x,y\in G\Rightarrow x\ i\ y\in G$.
          \item Sea $x\in G$, entonces $\exists F_1:x\in F_1$. Sea $y\in L:x\leq y$, como $x\in F_1\land x\leq y$, por def. de filtro $y\in F_1$. Como $F_1\subseteq G$, entonces $y\in G$. Por ello, $x\in G\wedge x\leq y\Rightarrow y\in G$.
          \item Debido a las tres propiedades anteriores, por def. tenemos que $G$ es un \textit{filtro}
          \item Como $x_0\notin F_1\forall F_1\in C$, claramente $x_0\notin G$
          \item Sea $F_1\in C$, entonces $F_1\in\mathcal{F}$, por lo que por def. $F\subseteq F_1$. Luego, como $F_1\subseteq G$, entonces $F\subseteq G$.
          \item Como $G$ es un filtro, $x_0\notin G$ y $F\subseteq G$, entonces $G\in\mathcal{F}$.
          \item Luego, llegamos a que $G$ es cota superior de $C$
        \end{itemize}
\end{itemize}

Ahora, como $(\mathcal{F},\subseteq)$ es un poset y toda cadena de $(\mathcal{F},\subseteq)$ tiene cota superior, entonces por el \textbf{Lema de Zorn}, hay un elemento maximal en $(\mathcal{F},\subseteq)$. Sea $P$ ese elemento maximal, vamos a ver que $P$ es un filtro primo.

Supongamos $x\ s\ y\in P$ y $x,y\notin P$. Como $[P\cup\{x\}),[P\cup\{y\})$ son filtros que por lema cumplen que contienen a $P\cup\{x\}$ y a $P\cup\{y\}$ respectivamente, y como $x,y\notin P$, entonces claramente $P\subsetneqq [P\cup\{x\}),[P\cup\{y\})$. Ahora, como $P$ es maximal, entonces $[P\cup\{x\}),[P\cup\{y\})\notin\mathcal{F}$, por lo que: o no son filtros, o contienen como elemento a $x_0$, o no contienen a $F$. De aquí llegamos a que $x_0\in [P\cup\{x\}),[P\cup\{y\})$.

Como $x_0\in[P\cup\{x\}),[P\cup\{y\})$, por definición de filtro generado tenemos que $\exists p_1,\dots,p_n\in P$ y $\exists q_1,\dots,q_m\in P$ tales que:
\begin{equation*}
  \begin{aligned}
    x_0 & \geq p_1\ i\ \dots\ i\ p_n\ i\ x \\
    x_0 & \geq q_1\ i\ \dots\ i\ q_m\ i\ y
  \end{aligned}
\end{equation*}
(Notar que colocamos a $x$ e $y$ porque sino fueran necesarios, $x_0\in [P)$ y sería absurdo ya que $[P)=P\in\mathcal{F}$ porque $P$ es maximal)

Sea $p=p_1\ i\ \dots\ i\ p_n\ i\ q_1\ i\ \dots\ i\ q_m$, tenemos que:
\begin{equation*}
  \begin{aligned}
    x_0 & \geq p\ i\ x \\
    x_0 & \geq p\ i\ y \\
  \end{aligned}
\end{equation*}
Luego, por propiedad de reticulado, $x_0\geq (p\ i\ x)\ s\ (p\ i\ y)$. Como el reticulado es distributivo, entonces $x_0\geq p\ i\ (x\ s\ y)$. Como $p,(x\ s\ y)\in P$, por def. de filtro $p\ i\ (x\ s\ y)\in P$. Ahora, como $P=[P)$, por def. de filtro generado tenemos que $x_0\in P$. Finalmente, como sabemos que $P\in\mathcal{F}$, entonces $x_0\notin P$ y llegamos a un absurdo que vino de suponer que $x\ s\ y\in P$ y $x,y\notin P$.

Por ello, tenemos que $x\ s\ y\in P\Rightarrow(x\in P\lor y\in P)$. Luego, por ello y dado que $P\neq L$ (porque $x_0\notin P$), tenemos que $P$ es un filtro primo por def.

Con todo esto, entonces, llegamos a que $P$ es un filtro primo tal que $x_0\notin P$ y $F\subseteq P$, por lo que se demuestra. $\blacksquare$

\section*{Propiedades básicas de la consistencia}
Sea $(\Sigma,\tau)$ una teoría:
\begin{enumerate}
  \item Si $(\Sigma,\tau)$ es inconsistente, entonces $(\Sigma,\tau)\vdash\varphi$ para toda sentencia $\varphi$
  \item Si $(\Sigma,\tau)$ es consistente y $(\Sigma,\tau)\vdash\varphi$, entonces $(\Sigma\cup\{\varphi\},\tau)$ es consistente
  \item Si $(\Sigma,\tau)\nvdash\neg\varphi$, entonces $(\Sigma\cup\{\varphi\},\tau)$ es consistente
\end{enumerate}
\subsection*{Demostración}
Demostremos cada punto por separado:
\begin{enumerate}
  \item Como $(\Sigma,\tau)$ es inconsistente, por def. $\exists\psi\in S^\tau:(\Sigma,\tau)\vdash(\psi\land\neg\psi)$. Sea $\psi$ esa sentencia, podemos ver que dada una sentencia $\varphi$ se cumple que $(\Sigma,\tau)\vdash\varphi$ y la prueba que lo atestigua es:
        \begin{equation*}
          \begin{alignedat}{2}
            1.\quad & \neg\varphi                      &  & \qquad\qquad\text{HIP1}           \\
            2.\quad & \psi\lor\neg\psi                 &  & \qquad\qquad\text{TESIS1 AXIPROP} \\
            3.\quad & \neg\varphi\to(\psi\lor\neg\psi) &  & \qquad\qquad\text{CONC}           \\
            4.\quad & \varphi                          &  & \qquad\qquad\text{ABS}(3)         \\
          \end{alignedat}
        \end{equation*}
        Con ello, se demuestra. $\blacksquare$
  \item Supongamos que $(\Sigma,\tau)$ es consistente y $(\Sigma,\tau)\vdash\varphi$. Si $(\Sigma\cup\{\varphi\},\tau)$ fuera inconsistente, por def. $\exists\psi\in S^\tau:(\Sigma\cup\{\varphi\},\tau)\vdash(\psi\land\neg\psi)$. Luego, sea $\psi$ esa sentencia, por lema de \textit{uso de teoremas}, llegamos a que $(\Sigma,\tau)\vdash(\psi\land\neg\psi)$, por lo que $(\Sigma,\tau)$ es inconsistente por def.. Como $(\Sigma,\tau)$ es consistente por suposición y llegamos a que es inconsistente, tenemos un absurdo que vino de suponer que $(\Sigma\cup\{\varphi\},\tau)$ es inconsistente. Luego, es consistente y se demuestra. $\blacksquare$
  \item Supongamos que $(\Sigma,\tau)\nvdash\neg\varphi$. Si $(\Sigma\cup\{\varphi\},\tau)$ fuera inconsistente, por def. $\exists\psi\in S^\tau:(\Sigma\cup\{\varphi\},\tau)\vdash(\psi\land\neg\psi)$. Sea $\psi$ esa sentencia, por \textit{lema} sabemos que eso implica que $(\Sigma,\tau)\vdash(\varphi\to(\psi\land\neg\psi))$. Ahora, como $\neg\varphi$ se deduce de $\varphi\to(\psi\land\neg\psi)$ por la regla del absurdo, llegamos a que $(\Sigma,\tau)\vdash\neg\varphi$. Con ello, llegamos a un absurdo que vino de suponer que $(\Sigma\cup\{\varphi\},\tau)$ era inconsistente. Luego, es consistente y se demuestra. $\blacksquare$
\end{enumerate}

\end{document}