\documentclass{article}
\usepackage[utf8]{inputenc}
\usepackage{geometry}
 \geometry{
 a4paper,
 total={170mm,257mm},
 left=20mm,
 top=10mm,
 }
\usepackage{graphicx}
\usepackage{titling}
\usepackage{lipsum}  
\usepackage{lmodern}
\usepackage{amssymb}
\usepackage{amsmath}

\title{Combo 4 de teoremas}
\author{Emanuel Nicolás Herrador}
\date{November 2024}
 
\makeatletter
\def\@maketitle{%
  \newpage
  \null
  \vskip 1em%
  \begin{center}%
  \let \footnote \thanks
    {\LARGE \@title \par}%
    \vskip 1em%
    {\large \@author\quad-\quad \@date}%
  \end{center}%
  \par
  \vskip 1em}
\makeatother

\begin{document}

\maketitle

\section*{Propiedades básicas de la deducción}
Sea $(\Sigma,\tau)$ una teoría:
\begin{enumerate}
  \item (Uso de Teoremas). Si $(\Sigma,\tau)\vdash\varphi_1,\dots,\varphi_n$ y $(\Sigma\cup\{\varphi_1,\dots,\varphi_n\},\tau)\vdash\varphi$, entonces $(\Sigma,\tau)\vdash\varphi$
  \item Supongamos $(\Sigma,\tau)\vdash\varphi_1,\dots,\varphi_n$. Si $R$ es una regla distinta de $GENERALIZACION$ y $ELECCION$, y $\varphi$ se deduce de $\varphi_1,\dots,\varphi_n$ por la regla $R$, entonces $(\Sigma,\tau)\vdash\varphi$
  \item $(\Sigma,\tau)\vdash(\varphi\to\psi)$ sii $(\Sigma\cup\{\varphi\},\tau)\vdash\psi$
\end{enumerate}
\subsection*{Demostración}
Vamos a usar los siguientes dos \textit{lemas} en la demostración:
\begin{itemize}
  \item \textit{Cambio de índice de hipótesis}: Sea $(\boldsymbol{\varphi},\mathbf{J})$ una prueba formal de $\varphi$ en $(\Sigma,\tau)$. Sea $m\in N$ tal que $\mathbf{J}_i\neq\text{HIPOTESIS}\bar{m}$ para cada $i=1,\dots,n(\boldsymbol{\varphi})$. Supongamos que $\mathbf{J}_i=\text{HIPOTESIS}\bar{k}$ y que $\mathbf{J}_j=\text{TESIS}\bar{k}\alpha$ con $[\alpha]_1\notin Num$. Sea $\tilde{\mathbf{J}}$ el resultado de reemplazar en $\mathbf{J}$ la justificación $\mathbf{J}_i$ por $\text{HIPOTESIS}\bar{m}$ y reemplazar la justificación $\mathbf{J}_j$ por $\text{TESIS}\bar{m}\alpha$. Entonces $(\boldsymbol{\varphi},\tilde{\mathbf{J}})$ es una prueba formal de $\varphi$ en $(\Sigma,\tau)$.
  \item \textit{Cambio de constantes auxiliares}: Sea $(\boldsymbol{\varphi},\mathbf{J})$ una prueba formal de $\varphi$ en $(\Sigma,\tau)$. Sea $\mathcal{C}_1$ el conjunto de nombres de constante que ocurren en $\boldsymbol{\varphi}$ y que no pertenecen a $\mathcal{C}$. Sea $e\in\mathcal{C}_1$. Sea $\tilde{e}\notin\mathcal{C}\cup\mathcal{C}_1$ tal que $(\mathcal{C}\cup(\mathcal{C}_1-\{e\})\cup\{\tilde{e}\},\mathcal{F},\mathcal{R},a)$ es un tipo. Sea $\tilde{\boldsymbol{\varphi}}_i=\text{resultado de reemplazar en }\boldsymbol{\varphi}_i\text{ cada ocurrencia de }e\text{ por }\tilde{e}$. Entonces $(\tilde{\boldsymbol{\varphi}}_1\dots\tilde{\boldsymbol{\varphi}}_{n(\boldsymbol{\varphi})},\mathbf{J})$ es una prueba formal de $\varphi$ en $(\Sigma,\tau)$.
\end{itemize}

\vspace{0.4cm}
A continuación, demostraremos cada uno de los puntos por separado.
\subsubsection*{Punto (1)}
Notemos que basta con hacer el caso $n=1$, porque si $n\geq 2$, entonces se obtiene aplicando $n$ veces el caso igual a $1$.

Supongamos entonces que $(\Sigma,\tau)\vdash\varphi_1$ y que $(\Sigma\cup\{\varphi_1\})\vdash\varphi$. Sea $(\alpha_1\dots\alpha_h,I_1\dots I_h)$ una prueba formal de $\varphi_1$ en $(\Sigma,\tau)$; y sea $(\psi_1\dots\psi_m,J_1\dots J_m)$ una prueba formal de $\varphi$ en $(\Sigma\cup\{\varphi_1\},\tau)$. Notemos que por los \textit{lemas} anteriores podemos suponer que las pruebas no comparten ningún nombre de constante auxiliar y que tampoco comparten números asociados a hipótesis o tesis.

Por ello, para cada $i=1,\dots,m$, definamos $\tilde{J}_i$ de la siguiente manera:
\begin{itemize}
  \item Si $J_i=\alpha\text{AXIOMAPROPIO}$ con $\alpha\in\{\varepsilon\}\cup\{\text{TESIS}\bar{k}:k\in N\}$ y $\psi_i=\varphi_1$, entonces $\tilde{J}_i=\alpha\text{EVOCACION}(\bar{h})$
  \item Si $J_i=\alpha R(\bar{l_1},\dots,\bar{l_k})$ con $\alpha\in\{\varepsilon\}\cup\{\text{TESIS}\bar{k}:k\in N\}$, entonces $\tilde{J}_i=\alpha R(\overline{l_1+h},\dots,\overline{l_k+h})$
  \item Sino, $\tilde{J}_i=J_i$
\end{itemize}

Luego, $(\alpha_1\dots\alpha_h\psi_1\dots\psi_m,I_1\dots I_h\tilde{J}_1\dots\tilde{J}_m)$ es una prueba formal de $\varphi$ en $(\Sigma,\tau)$, por lo que $(\Sigma,\tau)\vdash\varphi$ y se demuestra. $\blacksquare$

\subsubsection*{Punto (2)}
Notemos que:
\begin{equation*}
  \begin{alignedat}{2}
    1.     & \quad \varphi_1 &  & \qquad\text{AXIOMAPROPIO} \\
    2.     & \quad \varphi_2 &  & \qquad\text{AXIOMAPROPIO} \\
    \vdots & \quad \vdots    &  & \qquad\vdots              \\
    n.     & \quad \varphi_n &  & \qquad\text{AXIOMAPROPIO} \\
    n+1.   & \quad\varphi    &  & \qquad R(1,\dots,\bar{n})
  \end{alignedat}
\end{equation*}
es una prueba formal de $\varphi$ en $(\Sigma\cup\{\varphi_1,\dots,\varphi_n\},\tau)$, por lo que $(\Sigma\cup\{\varphi_1,\dots,\varphi_n\},\tau)\vdash\varphi$. Como suponemos $(\Sigma,\tau)\vdash\varphi_1,\dots,\varphi_n$, por el punto $\mathbf{(1)}$ tenemos que $(\Sigma,\tau)\vdash\varphi$ por lo que se demuestra. $\blacksquare$

\subsubsection*{Punto (3)}
Veamos los dos casos:
\begin{itemize}
  \item \textit{Ida}: Supongamos $(\Sigma,\tau)\vdash(\varphi\to\psi)$. Luego, claramente $(\Sigma\cup\{\varphi\},\tau)\vdash(\varphi\to\psi),\varphi$, por lo que por el punto $\mathbf{(2)}$ usando $\text{MODUSPONENS}$ tenemos que $(\Sigma\cup\{\varphi\},\tau)\vdash\psi$. Por ello, se demuestra la ida.
  \item \textit{Vuelta}: Supongamos $(\Sigma\cup\{\varphi\},\tau)\vdash\psi$. Sea $(\varphi_1\dots\varphi_n,J_1\dots J_n)$ una prueba formal de $\psi$ en $(\Sigma\cup\{\varphi\},\tau)$, entonces para cada $i=1,\dots,n$ definamos $\tilde{J}_i$ del siguiente modo:
        \begin{itemize}
          \item Si $J_i=\alpha\text{AXIOMAPROPIO}$ con $\alpha\in\{\varepsilon\}\cup\{\text{TESIS}\bar{k}:k\in N\}$ y $\varphi_i=\varphi$, entonces $\tilde{J}_i=\alpha\text{EVOCACION}(1)$
          \item Si $J_i=\alpha R(\bar{l_1},\dots,\bar{l_k})$ con $\alpha\in\{\varepsilon\}\cup\{\text{TESIS}\bar{k}:k\in N\}$, entonces $\tilde{J}_i=\alpha R(\overline{l_1+1},\dots,\overline{l_k+1})$
          \item Sino, $\tilde{J}_i=J_i$
        \end{itemize}
        Sea $m$ tal que ninguna $J_i$ es igual a $\text{HIPOTESIS}\bar{m}$. Entonces
        \begin{equation*}
          (\varphi\varphi_1\dots\varphi_n(\varphi\to\psi),\text{HIPOTESIS}\bar{m}\tilde{J}_1\dots\tilde{J}_{n-1}TESIS\bar{m}\tilde{J}_nCONCLUSION)
        \end{equation*}
        es una prueba formal de $(\varphi\to\psi)$ en $(\Sigma,\tau)$. Luego, $(\Sigma,\tau)\vdash(\varphi\to\psi)$ y se demuestra la vuelta.
\end{itemize}

Por ello, se demuestra el punto $\mathbf{(3)}$. $\blacksquare$

\section*{Teorema}
Sea $(L,s,i,\ ^c,0,1)$ un álgebra de Boole y sean $a,b\in B$. Se tiene que:
\begin{enumerate}
  \item $(a\ i\ b)^c=a^c\ s\ b^c$
  \item $a\ i\ b=0$ sii $b\leq a^c$
\end{enumerate}
\subsection*{Demostración}
Demostremos cada uno de los puntos por separado.

\subsubsection*{Punto (1)}
Vamos a usar el \textit{lema} que dice que: Si $(L,s,i,0,1)$ es un reticulado acotado y distributivo, entonces todo elemento tiene a lo sumo un complemento. Es decir, si $x\ s\ u=x\ s\ v=1$ y $x\ i\ u=x\ i\ v=0$, entonces $u=v$, cualesquiera sean $x,u,v\in L$.

\vspace{0.3cm}
Notemos que:
\begin{equation*}
  \begin{alignedat}{2}
    (a^c\ s\ b^c)\ s\ (a\ i\ b) & = (a^c\ s\ b^c\ s\ a)\ i\ (a^c\ s\ b^c\ s\ b) &  & \qquad\text{distributividad} \\
                                & =(1\ s\ b^c)\ i\ (a^c\ s\ 1)                                                    \\
                                & =1\ i\ 1                                                                        \\
                                & =1
    \\
    \\
    (a^c\ s\ b^c)\ i\ (a\ i\ b) & = (a^c\ i\ a\ i\ b)\ s\ (b^c\ i\ a\ i\ b)     &  & \qquad\text{distributividad} \\
                                & =(0\ i\ b)\ s\ (0\ i\ a)                                                        \\
                                & =0\ s\ 0                                                                        \\
                                & =0
  \end{alignedat}
\end{equation*}

Luego, por def. tenemos que $a^c\ s\ b^c$ es el complemento de $a\ i\ b$. Como $(L,s,i,\ ^c,0,1)$ es un Álgebra de Boole, por def. es también un reticulado acotado y distributivo. Luego, por el anterior \textit{lema} sabemos que es único el complemento.

Por ello, $(a\ i\ b)^c=a^c\ s\ b^c$ y se demuestra. $\blacksquare$

\subsubsection*{Punto (2)}
Para demostrarlo, veamos ambos lados de la doble implicación:
\begin{itemize}
  \item \textit{Ida}: Supongamos $a\ i\ b=0$. Con ello:
        \begin{equation*}
          \begin{alignedat}{2}
            b & =b\ i\ 1                                                              \\
              & =b\ i\ ((a\ i\ b)\ s\ (a\ i\ b)^c) &  & \qquad\text{def. complemento} \\
              & =b\ i\ (0\ s\ (a\ i\ b)^c)         &  & \qquad\text{supuesto}         \\
              & =b\ i\ (a\ i\ b)^c                                                    \\
              & =b\ i\ (a^c\ s\ b^c)               &  & \qquad\text{punto (1)}        \\
              & =(b\ i\ a^c)\ s\ (b\ i\ b^c)       &  & \qquad\text{distributividad}  \\
              & =(b\ i\ a^c)\ s\ 0                                                    \\
              & =b\ i\ a^c
          \end{alignedat}
        \end{equation*}
        Luego, como $b=b\ i\ a^c$, por def. alternativa del orden parcial $\leq$, tenemos que $b\leq a^c$ y se demuestra la ida.

  \item \textit{Vuelta}: Supongamos $b\leq a^c$. Por def. alternativa del orden parcial $\leq$, tenemos que $b\ i\ a^c=b$. Con ello, veamos que:
        \begin{equation*}
          \begin{alignedat}{2}
            a\ i\ b & =a\ i\ (b\ i\ a^c) &  & \qquad\text{reemplazando} \\
                    & =(a\ i\ a^c)\ i\ b                                \\
                    & =0\ i\ b                                          \\
                    & =0
          \end{alignedat}
        \end{equation*}
        Luego, llegamos a que $a\ i\ b=0$ y se demuestra la vuelta.
\end{itemize}

Por ello, se demuestra el punto $\mathbf{(2)}$. $\blacksquare$

\section*{Lema}
Sean $(L,s,i)$ y $(L',s',i')$ reticulados terna y sean $(L,\leq)$ y $(L',\leq')$ los posets asociados. Sea $F:L\to L'$ una función. Entonces $F$ es un isomorfismo de $(L,s,i)$ en $(L',s',i')$ sii $F$ es un isomorfismo de $(L,\leq)$ en $(L',\leq')$
\subsection*{Demostración}
Para la demostración, vamos a usar el siguiente \textit{lema}: Sean $(P,\leq)$ y $(P',\leq')$ posets y $F$ un isomorfismo de $(P,\leq)$ en $(P',\leq')$, entonces:
\begin{itemize}
  \item $\forall x,y,z\in P,\ z=\text{sup}\{x,y\}\iff F(z)=\text{sup}\{F(x),F(y)\}$
  \item $\forall x,y,z\in P,\ z=\text{inf}\{x,y\}\iff F(z)=\text{inf}\{F(x),F(y)\}$
\end{itemize}

\vspace{0.5cm}
Vamos a demostrar cada uno de los lados de la doble implicación por separado:
\begin{itemize}
  \item \textit{Ida}: Supongamos $F$ es un isomorfismo de $(L,s,i)$ en $(L',s',i')$. Por def. de isomorfismo, $F$ es biyectiva y $F,F^{-1}$ son homomorfismos. Por ello, podemos ver que, sean $x,y\in L$:
        \begin{equation*}
          \begin{aligned}
            x\leq y\overset{\text{def. }\leq}{\Rightarrow}y=x\ s\ y\overset{\text{def. homomorfismo}}{\Rightarrow}F(y)=F(x\ s\ y)=F(x)\ s'\ F(y)\overset{\text{def. de }\leq}{\Rightarrow}F(x)\leq' F(y) \\
          \end{aligned}
        \end{equation*}
        Con ello, llegamos a que $F$ es un homomorfismo de $(L,\leq)$ en $(L',\leq')$. Como $F$ es biyectiva y de forma análoga a la anterior podemos ver que $F^{-1}$ es un homomorfismo de $(L',\leq')$ en $(L,\leq)$, entonces $F$ es un isomorfismo de $(L,\leq)$ en $(L',\leq')$ y se demuestra la ida.
  \item \textit{Vuelta}: Supongamos $F$ es un isomorfismo de $(L,\leq)$ en $(L',\leq')$. Por ello, tenemos que:
        \begin{equation*}
          \begin{alignedat}{2}
            \forall x,y,z\in P,\ z=x\ s\ y & \iff F(z)=F(x)\ s\ F(y) &  & \qquad\text{por lema}                        \\
            \forall x,y\in P,\ F(x\ s\ y)  & =F(x)\ s'\ F(y)         &  & \qquad\text{usando la prop. con }\Rightarrow
          \end{alignedat}
        \end{equation*}
        Análogamente, llegamos también a que $\forall x,y\in P,\ F(x\ i\ y) = F(x)\ i'\ F(y)$. Por ello, $F$ es un homomorfismo de $(L,s,i)$ en $(L',s',i')$.

        Ahora, como $F$ es biyectiva y de forma análoga $F^{-1}$ es un homomorfismo de $(L',s',i')$ en $(L,s,i)$, por def. $F$ es un isomorfismo.

        Con ello, se demuestra la vuelta.
\end{itemize}

Por todo ello, entonces, se demuestra el lema. $\blacksquare$

\end{document}