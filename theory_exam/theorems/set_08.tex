\documentclass{article}
\usepackage[utf8]{inputenc}
\usepackage{geometry}
 \geometry{
 a4paper,
 total={170mm,257mm},
 left=20mm,
 top=10mm,
 }
\usepackage{graphicx}
\usepackage{titling}
\usepackage{lipsum}  
\usepackage{lmodern}
\usepackage{amssymb}
\usepackage{amsmath}
\usepackage{enumitem}

\title{Combo 8 de teoremas}
\author{Emanuel Nicolás Herrador}
\date{November 2024}
 
\makeatletter
\def\@maketitle{%
  \newpage
  \null
  \vskip 1em%
  \begin{center}%
  \let \footnote \thanks
    {\LARGE \@title \par}%
    \vskip 1em%
    {\large \@author\quad-\quad \@date}%
  \end{center}%
  \par
  \vskip 1em}
\makeatother

\begin{document}

\maketitle

\section*{Lema}
Supongamos que $F:\mathbf{A}\to\mathbf{B}$ es un isomorfismo. Sea $\varphi=_d\varphi(v_1,\dots,v_n)\in F^\tau$. Entonces:
\begin{equation*}
  \mathbf{A}\vDash\varphi[a_1,\dots,a_n]\text{ sii }\mathbf{B}\vDash\varphi[F(a_1),\dots,F(a_n)]
\end{equation*}
para cada $a_1,\dots,a_n\in A$
\subsection*{Demostración}
Vamos a usar los siguientes resultados en la demostración:
\begin{itemize}
  \item \textit{Convención notacional}: Una vez declarado $\varphi=_d\varphi(v_1,\dots,v_n)$, si $\mathbf{A}$ es un modelo de tipo $\tau$ y $a_1,\dots,a_n\in A$, entonces $\mathbf{A}\vDash\varphi[a_1,\dots,a_n]$ significará que $\mathbf{A}\vDash\varphi[\vec{b}]$ donde $\vec{b}$ es una asignación tal que a cada $v_i$ le asigna el valor $a_i$.
  \item \textit{Lema}: Supongamos que $F:\mathbf{A}\to\mathbf{B}$ es un isomorfismo. Sea $\varphi\in F^\tau$, entonces $\mathbf{A}\vDash\varphi[(a_1,a_2,\dots)]\iff\mathbf{B}\vDash\varphi[(F(a_1),F(a_2),\dots)]$ para cada $(a_1,a_2,\dots)\in A^N$. En particular, $\mathbf{A}$ y $\mathbf{B}$ satisfacen las mismas sentencias de tipo $\tau$.
\end{itemize}

\vspace{0.3cm}
Ahora, supongamos $F:\mathbf{A}\to\mathbf{B}$ isomorfismo, $\varphi=_d\varphi(v_1,\dots,v_n)\in F^\tau$ y $a_1,\dots,a_n\in A$ fijos pero arbitrarios. Entonces:
\begin{equation*}
  \begin{alignedat}{2}
    \mathbf{A}\vDash\varphi[a_1,a_2,\dots,a_n] & \iff\mathbf{A}\vDash\varphi[\vec{b}]\text{ con }\vec{b}\text{ tq a }v_i\text{ le asigna }a_i               &  & \qquad\text{Por Convención Notacional} \\
                                               & \iff\mathbf{B}\vDash\varphi[(F(b_1),F(b_2),\dots)]\text{ con }\vec{b}\text{ tq a }v_i\text{ le asigna }a_i &  & \qquad\text{Por Lema}                  \\
                                               & \iff\mathbf{B}\vDash\varphi[\vec{c}]\text{ con }\vec{c}\text{ tq a }v_i\text{ le asigna }F(a_i)                                                        \\
                                               & \iff\mathbf{B}\vDash\varphi[F(a_1),F(a_2),\dots,F(a_n)]                                                    &  & \qquad\text{Por Convención Notacional} \\
  \end{alignedat}
\end{equation*}

Luego, como eran fijos pero arbitrarios, se demuestra el lema para toda asignación. $\blacksquare$

\section*{Lema}
Sean $(P,\leq)$ y $(P',\leq')$ posets. Supongamos $F$ es un isomorfismo de $(P,\leq)$ en $(P',\leq')$:
\begin{enumerate}[label=(\alph*)]
  \item Para cada $S\subseteq P$ y cada $a\in P$, se tiene que $a$ es cota superior (resp. inferior) de $S$ sii $F(a)$ es cota superior (resp. inferior) de $F(S)$
  \item Para cada $S\subseteq P$, se tiene que existe $\text{sup}(S)$ sii existe $\text{sup}(F(S))$ y en el caso de que existan tales elementos se tiene que $F(\text{sup}(S))=\text{sup}(F(S))$
\end{enumerate}
Veamos cada punto por separado.

\subsubsection*{Punto $\mathbf{(a)}$}
Vemos la ida y la vuelta por separado. Ver la cota superior e inferior es análogo, por lo que demostramos solo la primera.
\begin{itemize}
  \item \textit{Ida}: Supongamos $a$ cota superior de $S$. Sea $x\in F(S)$ fijo pero arbitrario y $s\in S:x=F(s)$. Como $s\leq a$, por def. de isomorfismo, $x=F(s)\leq' F(a)$. Luego, como $x$ era fijo pero arbitrario, se tiene que $F(a)$ es cota superior de $F(S)$ y se demuestra la ida.
  \item \textit{Vuelta}: Supongamos $F(a)$ cota superior de $F(S)$ con $a\in S$. Sea $s\in S$ fijo pero arbitrario, como $F(s)\leq'F(a)$ entonces por def. de isomorfismo, $s=F^{-1}(F(s))\leq F^{-1}(F(a))=a$. Como $s$ era fijo pero arbitrario, se tiene que $a$ es cota superior de $S$ y se demuestra la vuelta.
\end{itemize}

Con ello, se demuestra la doble implicación para el caso de la cota superior. Como son análogos ambos casos de cota superior e inferior, se demuestra el punto $\textbf{(a)}$. $\blacksquare$

\subsubsection*{Punto $\mathbf{(b)}$}
Vemos la ida y la vuelta por separado:
\begin{itemize}
  \item \textit{Ida}: Supongamos que existe $\text{sup}(S)$. Por $\mathbf{(a)}$ sabemos que $F(\text{sup}(S))$ es cota superior de $F(S)$. Sea $b$ un elemento fijo pero arbitrario tal que es cota superior de $F(S)$, entonces por $\mathbf{(a)}$ $F^{-1}(b)$ es cota superior de $F^{-1}(F(S))=S$ por lo que $\text{sup}(S)\leq F^{-1}(b)$. Luego, con ello, por def. de isomorfismo, $F(\text{sup}(S))\leq' F(F^{-1}(b))=b$.

        Con ello, como $b$ era fijo pero arbitrario, se tiene que se cumple para todo $b$ cota superior de $F(S)$. Luego, por def., $F(\text{sup}(S))$ es supremo de $F(S)$ por lo que se demuestra la ida.
  \item \textit{Vuelta}: Es totalmente análogo al caso anterior, dado que $F$ es un isomorfismo.
\end{itemize}

Con ello, entonces se demuestra el punto $\textbf{(b)}$. $\blacksquare$

\end{document}