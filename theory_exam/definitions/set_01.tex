\documentclass{article}
\usepackage[utf8]{inputenc}
\usepackage{geometry}
 \geometry{
 a4paper,
 total={170mm,257mm},
 left=20mm,
 top=2mm,
 }
\usepackage{graphicx}
\usepackage{titling}
\usepackage{lipsum}  
\usepackage{lmodern}
\usepackage{amssymb}
\usepackage{amsmath}

\title{Combo 1 de definiciones y convenciones notacionales}
\author{Emanuel Nicolás Herrador}
\date{November 2024}
 
\makeatletter
\def\@maketitle{%
  \newpage
  \null
  \vskip 1em%
  \begin{center}%
  \let \footnote \thanks
    {\LARGE \@title \par}%
    \vskip 1em%
    {\large \@author\quad-\quad \@date}%
  \end{center}%
  \par
  \vskip 1em}
\makeatother

\begin{document}

\maketitle

\section{$n(\mathbf{J})$}
\begin{quote}
    Defina $n(\mathbf{J})$ (para $\mathbf{J}\in Just^+$)
\end{quote}
Por \textit{lema} sabemos que: Sea $\mathbf{J}\in Just^+$, hay únicos $n\geq 1$ y $J_1,\dots,J_n\in Just$ tales que $\mathbf{J}=J_1\dots J_n$.
\newline
Dada $\mathbf{J}\in Just^+$, usaremos $n(\mathbf{J})$ para denotar al único $n$ cuya existencia garantiza el lema anterior.

\section{Par adecuado de tipo $\tau$}
\begin{quote}
    Defina "par adecuado de tipo $\tau$" (no hace falta que defina cuando $\mathbf{J}\in Just^+$ es balanceada)
\end{quote}
Un par adecuado de tipo $\tau$ es un par $(\varphi,\mathbf{J})\in S^\tau\times Just^+$ tal que $n(\varphi)=n(\mathbf{J})$ y $\mathbf{J}$ es balanceada.

\section{$Mod_T(\varphi)$}
\begin{quote}
    Defina $Mod_T(\varphi)$
\end{quote}
Sea $T=(\Sigma,\tau)$ una teoría. Dada $\varphi\in S^\tau$ definamos $Mod_T(\varphi)=\{\mathbf{A}:\mathbf{A}\text{ es modelo de }T\text{ y }\mathbf{A}\vDash\varphi\}$

\section{$\mathbf{A}\vDash\varphi[a_1,\dots,a_n]$}
\begin{quote}
    Dados $\varphi=_d\varphi(v_1,\dots,v_n)$, $\mathbf{A}$ una estructura de tipo $\tau$ y $a_1,\dots,a_n\in A$, defina qué significa $\mathbf{A}\vDash\varphi[a_1,\dots,a_n]$ (i.e., convención notacional $4$)
\end{quote}
Dados $\varphi=_d\varphi(v_1,\dots,v_n)$, $\mathbf{A}$ una estructura de tipo $\tau$ y $a_1,\dots,a_n\in A$, entonces $\mathbf{A}\vDash\varphi[a_1,\dots,a_n]$ significará que $\mathbf{A}\vDash\varphi[\vec{b}]$ donde $\vec{b}$ es una asignación tal que a cada $v_i$ le asigna el valor $a_i$.
\newline
En general, $\mathbf{A}\nvDash\varphi[a_1,\dots,a_n]$ significará que no sucede $\mathbf{A}\vDash\varphi[a_1,\dots,a_n]$

\section{$(L,s,i,\ ^c,0,1)/\theta$}
\begin{quote}
    Defina $(L,s,i,\ ^c,0,1)/\theta$ ($\theta$ una congruencia del reticulado complementado $(L,s,i,\ ^c,0,1)$)
\end{quote}
Sea $(L,s,i,\ ^c,0,1)$ un reticulado complementado. Una congruencia sobre $(L,s,i,\ ^c,0,1)$ será una relación de equivalencia sobre $L$ la cual cumpla:
\begin{enumerate}
    \item $\theta$ es una congruencia sobre $(L,s,i,0,1)$
    \item $x/\theta=y/\theta$ implica $x^c/\theta=y^c/\theta$
\end{enumerate}
Las condiciones anteriores nos permiten definir sobre $L/\theta$ dos operaciones binarias $\tilde{s}$ e $\tilde{i}$, y una operación unaria $^{\tilde{c}}$ de la siguiente manera:
\begin{equation*}
    \begin{aligned}
        x/\theta\ \tilde{s}\ y/\theta & = (x\ s\ y)/\theta \\
        x/\theta\ \tilde{i}\ y/\theta & = (x\ i\ y)/\theta \\
        (x/\theta)^{\tilde{c}}        & = x^c/\theta
    \end{aligned}
\end{equation*}
La $6$-upla $(L/\theta,\tilde{s},\tilde{i},\ ^{\tilde{c}},0/\theta,1/\theta)$ es llamada el cociente de $(L,s,i,\ ^c,0,1)$ sobre $\theta$ y la denotaremos con $(L,s,i,\ ^c,0,1)/\theta$.

\end{document}