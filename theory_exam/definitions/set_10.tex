\documentclass{article}
\usepackage[utf8]{inputenc}
\usepackage{geometry}
 \geometry{
 a4paper,
 total={170mm,257mm},
 left=20mm,
 top=2mm,
 }
\usepackage{graphicx}
\usepackage{titling}
\usepackage{lipsum}  
\usepackage{lmodern}
\usepackage{amssymb}
\usepackage{amsmath}

\title{Combo 10 de definiciones y convenciones notacionales}
\author{Emanuel Nicolás Herrador}
\date{November 2024}
 
\makeatletter
\def\@maketitle{%
  \newpage
  \null
  \vskip 1em%
  \begin{center}%
  \let \footnote \thanks
    {\LARGE \@title \par}%
    \vskip 1em%
    {\large \@author\quad-\quad \@date}%
  \end{center}%
  \par
  \vskip 1em}
\makeatother

\begin{document}

\maketitle

\section{Tesis del bloque $\langle i,j\rangle$ en $(\varphi,\mathbf{J})$}
\begin{quote}
  Defina "tesis del bloque $\langle i,j\rangle$ en $(\varphi,\mathbf{J})$"
\end{quote}
Sea $(\varphi,\mathbf{J})$ un par adecuado de tipo $\tau$ y $\langle i,j\rangle\in\mathcal{B}^\mathbf{J}$. Entonces $\varphi_j$ será la tesis del bloque $\langle i,j\rangle$ en $(\varphi,\mathbf{J})$.

\section{Teoría de primer orden consistente}
\begin{quote}
  Defina cuándo una teoría de primer orden $(\Sigma,\tau)$ es consistente
\end{quote}
Una teoría $(\Sigma,\tau)$ será inconsistente cuando haya una sentencia $\varphi$ tal que $(\Sigma,\tau)\vdash(\varphi\land\neg\varphi)$.
\newline
Una teoría $(\Sigma,\tau)$ será consistente cuando no sea inconsistente.

\section{Prueba elemental $\varphi$ en $(\Sigma,\tau)$}
\begin{quote}
  Dada una teoría elemental $(\Sigma,\tau)$ y una sentencia elemental pura $\varphi$ de tipo $\tau$, defina "prueba elemental $\varphi$ en $(\Sigma,\tau)$"
\end{quote}
Dada una teoría elemental $(\Sigma,\tau)$ y una sentencia elemental $\varphi$ la cual no posea nombres de elementos fijos, una prueba elemental de $\varphi$ en $(\Sigma,\tau)$ será una prueba de $\varphi$ que posea las siguientes características:
\begin{enumerate}
  \item En la prueba se parte de una estructura de tipo $\tau$ fija pero arbitraria, en el sentido de que lo único que sabemos es que ella satisface los axiomas de $\Sigma$ (i.e., es un modelo de $(\Sigma,\tau)$).
        \newline
        Además, esta es la única información particular que podemos usar.
        \newline
        Notar que este punto nos garantiza que una prueba elemental de $\varphi$ en $(\Sigma,\tau)$ es una forma sólida de justificar que cualquier estructura de tipo $\tau$ que satisfaga los axiomas de $(\Sigma,\tau)$ también satisfará $\varphi$.
  \item Las deducciones en la prueba son muy simples y obvias de justificar con mínimas frases en castellano.
  \item En la escritura de la prueba, lo concerniente a la matemática misma se expresa usando solo sentencias elementales de tipo $\tau$.
\end{enumerate}

\end{document}