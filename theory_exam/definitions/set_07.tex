\documentclass{article}
\usepackage[utf8]{inputenc}
\usepackage{geometry}
 \geometry{
 a4paper,
 total={170mm,257mm},
 left=20mm,
 top=2mm,
 }
\usepackage{graphicx}
\usepackage{titling}
\usepackage{lipsum}  
\usepackage{lmodern}
\usepackage{amssymb}
\usepackage{amsmath}

\title{Combo 7 de definiciones y convenciones notacionales}
\author{Emanuel Nicolás Herrador}
\date{November 2024}
 
\makeatletter
\def\@maketitle{%
  \newpage
  \null
  \vskip 1em%
  \begin{center}%
  \let \footnote \thanks
    {\LARGE \@title \par}%
    \vskip 1em%
    {\large \@author\quad-\quad \@date}%
  \end{center}%
  \par
  \vskip 1em}
\makeatother

\begin{document}

\maketitle

\section{$v$ es sustituible por $w$ en $\varphi$}
\begin{quote}
  Defina recursivamente la relación "$v$ es sustituible por $w$ en $\varphi$"
\end{quote}
Definiremos recursivamente la relación "$v$ es sustituible por $w$ en $\varphi$" del siguiente modo:
\begin{enumerate}
  \item Si $\varphi$ es atómica, entonces $v$ es sustituible por $w$ en $\varphi$
  \item Si $\varphi=(\varphi_1\eta\varphi_2)$ con $\eta\in\{\land,\lor,\to,\leftrightarrow\}$, entonces $v$ es sustituible por $w$ en $\varphi$ sii $v$ es sustituible por $w$ en $\varphi_1$ y en $\varphi_2$
  \item Si $\varphi=\neg\varphi_1$, entonces $v$ es sustituible por $w$ en $\varphi$ sii $v$ es sustituible por $w$ en $\varphi_1$
  \item Si $\varphi=Qv\varphi_1$ con $Q\in\{\forall,\exists\}$, entonces $v$ es sustituible por $w$ en $\varphi$
  \item Si $\varphi=Qw\varphi_1$ con $Q\in\{\forall,\exists\}$ y $v\in Li(\varphi_1)$, entonces $v$ no es sustituible por $w$ en $\varphi$
  \item Si $\varphi=Qw\varphi_1$ con $Q\in\{\forall,\exists\}$ y $v\notin Li(\varphi_1)$, entonces $v$ es sustituible por $w$ en $\varphi$
  \item Si $\varphi=Qu\varphi_1$ con $Q\in\{\forall,\exists\}$, con $u\neq v$ y $u\neq w$, entonces $v$ es sustituible por $w$ en $\varphi$ sii $v$ es sustituible por $w$ en $\varphi_1$
\end{enumerate}

\section{$\mathbf{J}\in Just^+$ es balanceada}
\begin{quote}
  Defina cuándo $\mathbf{J}\in Just^+$ es balanceada (no hace falta que defina $\mathcal{B}^\mathbf{J}$)
\end{quote}
Diremos que $\mathbf{J}\in Just^+$ es balanceada si se dan las siguientes:
\begin{enumerate}
  \item Por cada $k\in N$ a lo sumo hay un $i$ tal que $\mathbf{J}_i=\text{HIPOTESIS}\bar{k}$ y a lo sumo hay un $j$ tal que $\mathbf{J}_j=\text{TESIS}\bar{k}\alpha$, con $\alpha\in JustBas$
  \item Si $\mathbf{J}_i=\text{HIPOTESIS}\bar{k}$, entonces hay un $l>i$ tal que $\mathbf{J}_l=\text{TESIS}\bar{k}\alpha$, con $\alpha\in JustBas$
  \item Si $\mathbf{J}_i=\text{TESIS}\bar{k}\alpha$, con $\alpha\in JustBas$, entonces hay un $l<i$ tal que $\mathbf{J}_l=\text{HIPOTESIS}\bar{k}$
  \item Si $B_1,B_2\in\mathcal{B}^\mathbf{J}$, entonces $B_1\cap B_2=\emptyset$ o $B_1\subseteq B_2$ o $B_2\subseteq B_1$
\end{enumerate}

\section{Filtro del reticulado terna $(L,s,i)$}
\begin{quote}
  Defina "filtro del reticulado terna $(L,s,i)$"
\end{quote}
Un filtro de un reticulado terna $(L,s,i)$ será un subconjunto $F\subseteq L$ tal que:
\begin{enumerate}
  \item $F\neq\emptyset$
  \item $x,y\in F\Rightarrow x\ i\ y\in F$
  \item $x\in F$ y $x\leq y\Rightarrow y\in F$
\end{enumerate}

\section{Teoría elemental}
\begin{quote}
  Defina "teoría elemental"
\end{quote}
Una teoría elemental será un par $(\Sigma,\tau)$ tal que $\tau$ es un tipo cualquiera y $\Sigma$ es un conjunto de sentencias elementales de tipo $\tau$, las cuales no tienen nombres de elementos fijos.
\newline
Un modelo de $(\Sigma,\tau)$ será una estructura de tipo $\tau$ la cual haga verdaderos a todos los elementos de $\Sigma$.

\end{document}