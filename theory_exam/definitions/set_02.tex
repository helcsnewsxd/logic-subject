\documentclass{article}
\usepackage[utf8]{inputenc}
\usepackage{geometry}
 \geometry{
 a4paper,
 total={170mm,257mm},
 left=20mm,
 top=2mm,
 }
\usepackage{graphicx}
\usepackage{titling}
\usepackage{lipsum}  
\usepackage{lmodern}
\usepackage{amssymb}
\usepackage{amsmath}

\title{Combo 2 de definiciones y convenciones notacionales}
\author{Emanuel Nicolás Herrador}
\date{November 2024}
 
\makeatletter
\def\@maketitle{%
  \newpage
  \null
  \vskip 1em%
  \begin{center}%
  \let \footnote \thanks
    {\LARGE \@title \par}%
    \vskip 1em%
    {\large \@author\quad-\quad \@date}%
  \end{center}%
  \par
  \vskip 1em}
\makeatother

\begin{document}

\maketitle

\section{$(\Sigma,\tau)\vDash\varphi$}
\begin{quote}
  Defina $(\Sigma,\tau)\vDash\varphi$
\end{quote}
Dada $(\Sigma,\tau)$ una teoría, escribiremos $(\Sigma,\tau)\vDash\varphi$ cuando $\varphi$ sea verdadera en todo modelo de $(\Sigma,\tau)$.

\section{Partición de $A$, $R_\mathcal{P}$}
\begin{quote}
  Defina "partición de $A$" y $R_\mathcal{P}$
\end{quote}
Dado un conjunto $A$, por una partición de $A$ entenderemos un conjunto $\mathcal{P}$ tal que:
\begin{itemize}
  \item Cada elemento de $\mathcal{P}$ es un subconjunto no vacío de $A$
  \item Si $S_1,S_2\in\mathcal{P}$ y $S_1\neq S_2$, entonces $S_1\cap S_2=\emptyset$
  \item $A=\{a:a\in S,\text{ para algún }S\in\mathcal{P}\}$.
\end{itemize}
Dada una partición $\mathcal{P}$ de un conjunto $A$ podemos definir una relación binaria asociada a $\mathcal{P}$ de la siguiente manera:
\begin{equation*}
  R_\mathcal{P}=\{(a,b)\in A^2:a,b\in S,\text{ para algún }S\in\mathcal{P}\}
\end{equation*}

\section{$\varphi_i$ está bajo la hipótesis $\varphi_l$ en $(\varphi,\mathbf{J})$}
\begin{quote}
  Defina cuándo "$\varphi_i$ está bajo la hipótesis $\varphi_l$ en $(\varphi,\mathbf{J})$" (no hace falta que defina $\mathcal{B}^\mathbf{J}$)
\end{quote}
Sea $(\varphi,\mathbf{J})$ un par adecuado de tipo $\tau$, diremos que $\varphi_i$ está bajo la hipótesis $\varphi_l$ en $(\varphi,\mathbf{J})$ cuando haya en $\mathcal{B}^\mathbf{J}$ un bloque de la forma $\langle l,j\rangle$ el cual contenga a $i$.

\section{$(L,s,i)/\theta$}
\begin{quote}
  Defina $(L,s,i)/\theta$ ($\theta$ una congruencia del reticulado terna $(L,s,i)$). No hace falta que defina el concepto de congruencia.
\end{quote}
Sea $(L,s,i)$ un reticulado terna. Una congruencia sobre $(L,s,i)$ será una relación de equivalencia $\theta$ sobre $L$ la cual cumpla:
\begin{enumerate}
  \item $x\theta x'$ y $y\theta y'$ implica $(x\ s\ y)\theta(x'\ s\ y')$ y $(x\ i\ y)\theta(x'\ i\ y')$
\end{enumerate}
Gracias a esta condición, podemos definir en forma inambigua sobre $L/\theta$ dos operaciones binarias $\tilde{s}$ e $\tilde{i}$, de la siguiente manera:
\begin{equation*}
  \begin{aligned}
    x/\theta\ \tilde{s}\ y/\theta & = (x\ s\ y)/\theta \\
    x/\theta\ \tilde{i}\ y/\theta & = (x\ i\ y)/\theta
  \end{aligned}
\end{equation*}
La terna $(L/\theta,\tilde{s},\tilde{i})$ es llamada el cociente de $(L,s,i)$ sobre $\theta$ y la denotamos con $(L,s,i)/\theta$.

\end{document}