\documentclass{article}
\usepackage[utf8]{inputenc}
\usepackage{geometry}
 \geometry{
 a4paper,
 total={170mm,257mm},
 left=20mm,
 top=2mm,
 }
\usepackage{graphicx}
\usepackage{titling}
\usepackage{lipsum}  
\usepackage{lmodern}
\usepackage{amssymb}
\usepackage{amsmath}

\title{Combo 6 de definiciones y convenciones notacionales}
\author{Emanuel Nicolás Herrador}
\date{November 2024}
 
\makeatletter
\def\@maketitle{%
  \newpage
  \null
  \vskip 1em%
  \begin{center}%
  \let \footnote \thanks
    {\LARGE \@title \par}%
    \vskip 1em%
    {\large \@author\quad-\quad \@date}%
  \end{center}%
  \par
  \vskip 1em}
\makeatother

\begin{document}

\maketitle

\section{Notación declaratoria para fórmulas}
\begin{quote}
  Explique la notación declaratoria para fórmulas con sus $3$ convenciones notacionales (convenciones $3,4,6$ de la guía $11$). Puede asumir la notación declaratoria para términos.
\end{quote}
Si $\varphi$ es una fórmula de tipo $\tau$, entonces escribiremos $\varphi=_d\varphi(v_1,\dots,v_n)$ para declarar que $v_1,\dots,v_n$ son variables distintas (con $n\geq 1$) y tales que $Li(\varphi)\subseteq\{v_1,\dots,v_n\}$.
\newline
El uso de declaraciones de la forma $\varphi=_d\varphi(v_1,\dots,v_n)$ será muy útil cuando se combina con ciertas convenciones notacionales que describiremos a continuación:
\begin{itemize}
  \item \textit{Convención 3}: Cuando hayamos hecho la declaración $\varphi=_d\varphi(v_1,\dots,v_n)$, si $P_1,\dots,P_n$ son palabras cualesquiera, entonces $\varphi(P_1,\dots,P_n)$ denotará la palabra que resulta de reemplazar (simultáneamente) cada ocurrencia libre de $v_1$ en $\varphi$ por $P_1$, cada ocurrencia libre $v_2$ en $\varphi$ por $P_2$, etc.
        \newline
        Notar que cuando las palabras $P_i$ son términos, entonces $\varphi(P_1,\dots,P_n)$ es una fórmula.
  \item \textit{Convención 4}: Cuando hayamos declarado $\varphi=_d\varphi(v_1,\dots,v_n)$, si $\mathbf{A}$ es un modelo de tipo $\tau$ y $a_1,\dots,a_n\in A$, entonces $\mathbf{A}\vDash\varphi[a_1,\dots,a_n]$ significará que $\mathbf{A}\vDash\varphi[\vec{b}]$, donde $\vec{b}$ es una asignación tal que a cada $v_i$ le asigna el valor $a_i$.
        \newline
        En general, $\mathbf{A}\nvDash\varphi[a_1,\dots,a_n]$ significará que no sucede $\mathbf{A}\vDash\varphi[a_1,\dots,a_n]$
  \item \textit{Convención 6}: Cuando hayamos declarado $\varphi=_d\varphi(v_1,\dots,v_n)$, entonces:
        \begin{itemize}
          \item Si $\varphi=(t\equiv s)$, con $t,s\in T^\tau$ únicos, supondremos tácitamente que también hemos hecho las declaraciones $t=_d t(v_1,\dots,v_n)$ y $s=_d s(v_1,\dots,v_n)$
          \item Si $\varphi=r(t_1,\dots,t_m)$, con $r\in\mathcal{R}_m$ y $t_1,\dots,t_m\in T^\tau$ únicos, supondremos tácitamente que también hemos hecho las declaraciones $t_1=_d t_1(v_1,\dots,v_n),\dots,t_m=_d t_m(v_1,\dots,v_n)$
          \item Si $\varphi=(\varphi_1\eta\varphi_2)$, con $\eta\in\{\land,\lor,\to,\leftrightarrow\}$ y $\varphi_1,\varphi_2\in F^\tau$ únicas, supondremos tácitamente que también hemos hecho las declaraciones $\varphi_1=_d\varphi_1(v_1,\dots,v_n)$ y $\varphi_2=_d\varphi_2(v_1,\dots,v_n)$
          \item Si $\varphi=\neg\varphi_1$, con $\varphi_1\in F^\tau$ única, supondremos tácitamente que también hemos hecho la declaración $\varphi_1=_d\varphi_1(v_1,\dots,v_n)$
          \item Si $\varphi=Qv_j\varphi_1$, con $Q\in\{\forall,\exists\}$, $v_j\in\{v_1,\dots,v_n\}$ y $\varphi_1\in F^\tau$ únicas, supondremos tácitamente que también hemos hecho la declaración $\varphi_1=_d\varphi_1(v_1,\dots,v_n)$
          \item Si $\varphi=Qv\varphi_1$, con $Q\in\{\forall,\exists\}$, $v\in Var-\{v_1,\dots,v_n\}$ y $\varphi_1\in F^\tau$ únicas, supondremos tácitamente que también hemos hecho la declaración $\varphi_1=_d\varphi_1(v_1,\dots,v_n,v)$
        \end{itemize}
\end{itemize}

\end{document}