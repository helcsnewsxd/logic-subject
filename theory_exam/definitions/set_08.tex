\documentclass{article}
\usepackage[utf8]{inputenc}
\usepackage{geometry}
 \geometry{
 a4paper,
 total={170mm,257mm},
 left=20mm,
 top=2mm,
 }
\usepackage{graphicx}
\usepackage{titling}
\usepackage{lipsum}  
\usepackage{lmodern}
\usepackage{amssymb}
\usepackage{amsmath}

\title{Combo 8 de definiciones y convenciones notacionales}
\author{Emanuel Nicolás Herrador}
\date{November 2024}
 
\makeatletter
\def\@maketitle{%
  \newpage
  \null
  \vskip 1em%
  \begin{center}%
  \let \footnote \thanks
    {\LARGE \@title \par}%
    \vskip 1em%
    {\large \@author\quad-\quad \@date}%
  \end{center}%
  \par
  \vskip 1em}
\makeatother

\begin{document}

\maketitle

\section{$(L,s,i,\ ^c,0,1)/\theta$}
\begin{quote}
  Defina $(L,s,i,\ ^c,0,1)/\theta$ ($\theta$ una congruencia del reticulado complementado $(L,s,i,\ ^c,0,1)$)
\end{quote}
Sea $(L,s,i,\ ^c,0,1)$ un reticulado complementado. Una congruencia sobre $(L,s,i,\ ^c,0,1)$ será una relación de equivalencia sobre $L$ la cual cumpla:
\begin{enumerate}
  \item $\theta$ es una congruencia sobre $(L,s,i,0,1)$
  \item $x/\theta=y/\theta$ implica $x^c/\theta=y^c/\theta$
\end{enumerate}
Las condiciones anteriores nos permiten definir sobre $L/\theta$ dos operaciones binarias $\tilde{s}$ e $\tilde{i}$, y una operación unaria $^{\tilde{c}}$ de la siguiente manera:
\begin{equation*}
  \begin{aligned}
    x/\theta\ \tilde{s}\ y/\theta & = (x\ s\ y)/\theta \\
    x/\theta\ \tilde{i}\ y/\theta & = (x\ i\ y)/\theta \\
    (x/\theta)^{\tilde{c}}        & = x^c/\theta
  \end{aligned}
\end{equation*}
La $6$-upla $(L/\theta,\tilde{s},\tilde{i},\ ^{\tilde{c}},0/\theta,1/\theta)$ es llamada el cociente de $(L,s,i,\ ^c,0,1)$ sobre $\theta$ y la denotaremos con $(L,s,i,\ ^c,0,1)/\theta$.

\section{$\mathbf{A}\vDash\varphi[a_1,\dots,a_n]$}
\begin{quote}
  Dados $\varphi=_d\varphi(v_1,\dots,v_n)$, $\mathbf{A}$ una estructura de tipo $\tau$ y $a_1,\dots,a_n\in A$, defina qué significa $\mathbf{A}\vDash\varphi[a_1,\dots,a_n]$ (i.e., convención notacional $4$)
\end{quote}
Dados $\varphi=_d\varphi(v_1,\dots,v_n)$, $\mathbf{A}$ una estructura de tipo $\tau$ y $a_1,\dots,a_n\in A$, entonces $\mathbf{A}\vDash\varphi[a_1,\dots,a_n]$ significará que $\mathbf{A}\vDash\varphi[\vec{b}]$ donde $\vec{b}$ es una asignación tal que a cada $v_i$ le asigna el valor $a_i$.
\newline
En general, $\mathbf{A}\nvDash\varphi[a_1,\dots,a_n]$ significará que no sucede $\mathbf{A}\vDash\varphi[a_1,\dots,a_n]$

\section{Supremo de $S$ en $(P,\leq)$}
\begin{quote}
  Dado un poset $(P,\leq)$, defina "$a$ es supremo de $S$ en $(P,\leq)$"
\end{quote}
Sea $(P,\leq)$ un poset. Dado $S\subseteq P$, diremos que un elemento $a\in P$ es cota superior de $S$ en $(P,\leq)$ cuando $\forall b\in S,\ b\leq a$.
\newline
Un elemento $a\in P$ será llamado supremo de $S$ en $(P,\leq)$ cuando se den las siguientes dos propiedades:
\begin{enumerate}
  \item $a$ es cota superior de $S$ en $(P,\leq)$
  \item $\forall b\in P$, si $b$ es una cota superior de $S$ en $(P,\leq)$, entonces $a\leq b$
\end{enumerate}

\section{$i$ es anterior a $j$ en $(\varphi,\mathbf{J})$}
\begin{quote}
  Defina "$i$ es anterior a $j$ en $(\varphi,\mathbf{J})$" (no hace falta que defina $\mathcal{B}^\mathbf{J}$)
\end{quote}
Sea $(\varphi,\mathbf{J})$ un par adecuado de tipo $\tau$, y sean $i,j\in\langle 1,n(\varphi)\rangle$. Diremos que $i$ es anterior a $j$ en $(\varphi,\mathbf{J})$ si $i<j$ y además $\forall B\in\mathcal{B}^\mathbf{J},\ (i\in B\Rightarrow j\in B)$.

\end{document}