\documentclass{article}
\usepackage[utf8]{inputenc}
\usepackage{geometry}
 \geometry{
 a4paper,
 total={170mm,257mm},
 left=20mm,
 top=10mm,
 }
\usepackage{graphicx}
\usepackage{titling}
\usepackage{lipsum}  
\usepackage{lmodern}
\usepackage{amssymb}
\usepackage{amsmath}

\title{Combo 9 de definiciones y convenciones notacionales}
\author{Emanuel Nicolás Herrador}
\date{November 2024}
 
\makeatletter
\def\@maketitle{%
  \newpage
  \null
  \vskip 1em%
  \begin{center}%
  \let \footnote \thanks
    {\LARGE \@title \par}%
    \vskip 1em%
    {\large \@author\quad-\quad \@date}%
  \end{center}%
  \par
  \vskip 1em}
\makeatother

\begin{document}

\maketitle

\section{Término elemental de tipo $\tau$}
\begin{quote}
  Defina "término elemental de tipo $\tau$"
\end{quote}
Dado un tipo $\tau=(\mathcal{C},\mathcal{F},\mathcal{R},a)$, definimos los términos elementales de tipo $\tau$ del siguiente modo:
\begin{itemize}
  \item Cada palabra de $\mathcal{C}$ es un término elemental de tipo $\tau$
  \item Cada variable es un término elemental de tipo $\tau$
  \item Cada nombre de elemento fijo es un término elemental de tipo $\tau$
  \item Si $f\in\mathcal{F}_n$, con $n\geq 1$ y $t_1,\dots,t_n$ términos elementales de tipo $\tau$, entonces $f(t_1,\dots,t_n)$ es un término elemental de tipo $\tau$
  \item Una palabra es un término elemental de tipo $\tau$ sii se puede construir usando las cláusulas anteriores
\end{itemize}

\section{$\dashv\vdash_T$}
\begin{quote}
  Defina $\dashv\vdash_T$
\end{quote}
Sea $T=(\Sigma,\tau)$ una teoría, podemos definir la siguiente relación binaria sobre $S^\tau$:
\begin{equation*}
  \varphi\dashv\vdash_T\psi\text{ sii }T\vdash(\varphi\leftrightarrow\psi)
\end{equation*}
Es decir, $\dashv\vdash_T$ es una relación de equivalencia dada por:
\begin{equation*}
  \dashv\vdash_T=\{(\varphi,\psi)\in S^\tau:T\vdash(\varphi\leftrightarrow\psi)\}
\end{equation*}

\section{$s^T$}
\begin{quote}
  Defina $s^T$ (explique por qué la definición es inambigua)
\end{quote}
Dada una teoría $T=(\Sigma,\tau)$, definiremos sobre $S^\tau/\dashv\vdash_T$ la siguiente operación binaria $s^T$:
\begin{equation*}
  [\varphi]_T\ s^T\ [\psi]_T = [(\varphi\lor\psi)]_T
\end{equation*}
Para mostrar que es inambigua, debemos demostrar la siguiente propiedad:
\begin{equation*}
  ([\varphi]_T=[\varphi']_T\text{ y }[\psi]_T=[\psi']_T)\Rightarrow[(\varphi\lor\psi)]_T=[(\varphi'\lor\psi')]_T
\end{equation*}
Es decir, debemos probar que:
\begin{equation*}
  (T\vdash(\varphi\leftrightarrow\varphi')\text{ y }T\vdash(\psi\leftrightarrow\psi'))\Rightarrow T\vdash((\varphi\lor\psi)\leftrightarrow(\varphi'\lor\psi'))
\end{equation*}
Para ello, sea $(\Sigma\cup\{(\varphi\leftrightarrow\varphi'),(\psi\leftrightarrow\psi')\},\tau)$, podemos considerar la siguiente prueba formal:
\begin{equation*}
  \begin{alignedat}{2}
    1. & \quad (\varphi\leftrightarrow\varphi')                      &  & \qquad \text{AXIOMAPROPIO}   \\
    2. & \quad (\psi\leftrightarrow\psi')                            &  & \qquad \text{AXIOMAPROPIO}   \\
    3. & \quad ((\varphi\lor\psi)\leftrightarrow(\varphi\lor\psi))   &  & \qquad \text{AXIOMALOGICO}   \\
    4. & \quad ((\varphi\lor\psi)\leftrightarrow(\varphi'\lor\psi))  &  & \qquad \text{REEMPLAZO}(1,3) \\
    5. & \quad ((\varphi\lor\psi)\leftrightarrow(\varphi'\lor\psi')) &  & \qquad \text{REEMPLAZO}(2,4) \\
  \end{alignedat}
\end{equation*}
Con ello, se atestigua que $(\Sigma\cup\{(\varphi\leftrightarrow\varphi'),(\psi\leftrightarrow\psi')\},\tau)\vdash((\varphi\lor\psi)\leftrightarrow(\varphi'\lor\psi'))$.
\newline
Finalmente, entonces, se demuestra que $s^T$ es inambigua.

\section{$\mathcal{A}_T$}
\begin{quote}
  Defina $\mathcal{A}_T$
\end{quote}
Dada una teoría $T=(\Sigma,\tau)$, denotaremos con $\mathcal{A}_T$ al álgebra de Boole $(S^\tau/\dashv\vdash_T,s^T,i^T,\ ^{c^T},0^T,1^T)$. El álgebra $\mathcal{A}_T$ será llamada el álgebra de Lindenbaum de la teoría $T$.
\newline
Notar que:
\begin{itemize}
  \item $[\varphi]_T\ s^T\ [\psi]_T = [(\varphi\lor\psi)]_T$
  \item $[\varphi]_T\ i^T\ [\psi]_T = [(\varphi\land\psi)]_T$
  \item $([\varphi]_T)^{c^T}=[\neg\varphi]_T$
  \item $0^T=\{\varphi\in S^\tau:\varphi\text{ es refutable en }T\}$ ($\varphi$ es refutable en $T$ si $T\vdash\neg\varphi$)
  \item $1^T=\{\varphi\in S^\tau:\varphi\text{ es un teorema de }T\}$
\end{itemize}

\section{Subuniverso de un reticulado complementado}
\begin{quote}
  Defina "$S$ es un subuniverso del reticulado complementado $(L,s,i,\ ^c,0,1)$"
\end{quote}
Sea $(L,s,i,\ ^c,0,1)$ un reticulado complementado. Un conjunto $S\subseteq L$ es llamado subuniverso de $(L,s,i,\ ^c,0,1)$ si $0,1\in S$ y además $S$ es cerrado bajo las operaciones $s,i,\ ^c$.

\end{document}