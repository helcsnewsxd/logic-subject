\documentclass{article}
\usepackage[utf8]{inputenc}
\usepackage{geometry}
 \geometry{
 a4paper,
 total={170mm,257mm},
 left=20mm,
 top=2mm,
 }
\usepackage{graphicx}
\usepackage{titling}
\usepackage{lipsum}  
\usepackage{lmodern}
\usepackage{amssymb}
\usepackage{amsmath}

\title{Combo 11 de definiciones y convenciones notacionales}
\author{Emanuel Nicolás Herrador}
\date{November 2024}
 
\makeatletter
\def\@maketitle{%
  \newpage
  \null
  \vskip 1em%
  \begin{center}%
  \let \footnote \thanks
    {\LARGE \@title \par}%
    \vskip 1em%
    {\large \@author\quad-\quad \@date}%
  \end{center}%
  \par
  \vskip 1em}
\makeatother

\begin{document}

\maketitle

\section{Programa de Lógica Matemática}
\begin{quote}
  Enuncie el programa de Lógica Matemática dado al final de la guía $8$ y explique brevemente con qué definiciones matemáticas se van resolviendo los tres primeros puntos y qué teoremas garantizan la resolución del 4to punto de dicho programa.
\end{quote}
El programa de lógica matemática dado es el siguiente:
\begin{enumerate}
  \item Dar un modelo matemático del concepto de fórmula elemental de tipo $\tau$
        \begin{itemize}
          \item \textit{Variables}
          \item \textit{Términos y subtérminos + Unicidad de la lectura de términos + Ocurrencia y reemplazos}
          \item \textit{Fórmulas y subfórmulas + Unicidad de la lectura de fórmulas + Ocurrencias}
          \item \textit{Variables libres + Ocurrencias}
        \end{itemize}
  \item Dar una definición matemática de cuándo una fórmula elemental de tipo $\tau$ es verdadera en una estructura de tipo $\tau$ para una asignación dada de valores a las variables libres y a los nombres de constantes fijas de la fórmula
        \begin{itemize}
          \item \textit{Asignación + Valor de un término en una estructura para una asignación + Reemplazo} $\downarrow^a_i$
          \item \textit{Relación} $\vDash$
        \end{itemize}
  \item Dar un modelo matemático del concepto de prueba elemental en una teoría elemental de tipo $\tau$. A estos objetos matemáticos los llamaremos pruebas formales de tipo $\tau$
        \begin{itemize}
          \item \textit{Notación declaratoria de términos y fórmulas}
          \item \textit{Teoría de primer orden}
          \item \textit{Axiomas propios + modelos}
          \item \textit{Reglas (Part, Exist, Evoc, Absur, ConjElim, EquivElim, DisjInt, Conm, ModPon, ConjInt, EquivInt, DisjElim, DivPorCas, Reemp, Trans, Generaliz, Elec)}
          \item \textit{Axiomas lógicos}
          \item \textit{Justificaciones básicas + Justificaciones + Bloques + Concatenaciones balanceadas de justificaciones}
          \item \textit{Pares adecuados + Hipótesis y tesis + Dependencia de constantes en pares adecuados}
          \item \textit{Prueba formal + Teorema}
        \end{itemize}
  \item Intenta probar matemáticamente que nuestro concepto de prueba formal de tipo $\tau$ es una correcta modelización matemática de la idea intuitiva de prueba elemental en una teoría elemental de tipo $\tau$
        \begin{itemize}
          \item \textit{Teorema de Corrección}: $(\Sigma,\tau)\vdash\varphi$ implica $(\Sigma,\tau)\vDash\varphi$
          \item \textit{Teorema de Completitud}: $(\Sigma,\tau)\vDash\varphi$ implica $(\Sigma,\tau)\vdash\varphi$
        \end{itemize}
\end{enumerate}

\end{document}