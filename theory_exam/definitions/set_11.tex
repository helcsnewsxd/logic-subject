\documentclass{article}
\usepackage[utf8]{inputenc}
\usepackage{geometry}
 \geometry{
 a4paper,
 total={170mm,257mm},
 left=20mm,
 top=2mm,
 }
\usepackage{graphicx}
\usepackage{titling}
\usepackage{lipsum}  
\usepackage{lmodern}
\usepackage{amssymb}
\usepackage{amsmath}

\title{Combo 11 de definiciones y convenciones notacionales}
\author{Emanuel Nicolás Herrador}
\date{November 2024}
 
\makeatletter
\def\@maketitle{%
  \newpage
  \null
  \vskip 1em%
  \begin{center}%
  \let \footnote \thanks
    {\LARGE \@title \par}%
    \vskip 1em%
    {\large \@author\quad-\quad \@date}%
  \end{center}%
  \par
  \vskip 1em}
\makeatother

\begin{document}

\maketitle

\section{Programa de Lógica Matemática}
\begin{quote}
  Enuncie el programa de Lógica Matemática dado al final de la guía $8$ y explique brevemente con qué definiciones matemáticas se van resolviendo los tres primeros puntos y qué teoremas garantizan la resolución del 4to punto de dicho programa.
\end{quote}
El programa de lógica matemática dado es el siguiente:
\begin{enumerate}
  \item Dar un modelo matemático del concepto de fórmula elemental de tipo $\tau$
        \begin{itemize}
          \item Sea $\tau=(\mathcal{C},\mathcal{F},\mathcal{R},a)$ un tipo, diremos que $\tau'$ es una \textit{extensión de} $\tau$ \textit{por nombres de constante} si $\tau'$ es de la forma $(\mathcal{C}',\mathcal{F},\mathcal{R},a)$ con $C'$ tal que $C\subseteq C'$
          \item Las fórmulas de tipo $\tau$ son un modelo matemático de las fórmulas elementales puras de tipo $\tau$ (i.e., sin nombres de elementos fijos)
          \item Los nombres de elementos fijos se usan en las pruebas elementales para denotar elementos fijos (a veces arbitrarios y otras veces que cumplen alguna propiedad)
          \item Cuando uno realiza una prueba elemental en una teoría elemental $(\Sigma,\tau)$, comienza la misma imaginando una estructura de tipo $\tau$ de la cual lo único que sabe es que cumple las sentencias de $\Sigma$. Luego, cuando fija un elemento y le pone nombre, digamos $b$, podemos pensar que expandió su estructura imaginaria a una de tipo $(\mathcal{C}\cup\{b\},\mathcal{F},\mathcal{R},a)$ y continúa su razonamiento. Esto lo podemos hacer muchas veces a lo largo de una prueba, por lo cual su estructura imaginaria va cambiando de tipo
          \item Esta mecánica de prueba nos deja ver que es natural modelizar las fórmulas elementales de tipo $\tau$ con fórmulas de tipo $\tau_1$, donde $\tau_1$ es alguna extensión de $\tau$ por nombres de constante
        \end{itemize}
  \item Dar una definición matemática de cuándo una fórmula elemental de tipo $\tau$ es verdadera en una estructura de tipo $\tau$ para una asignación dada de valores a las variables libres y a los nombres de constantes fijas de la fórmula
        \begin{itemize}
          \item La definición matemática de la relación "$\mathbf{A}\vDash\varphi[\vec{a}]$"
        \end{itemize}
  \item Dar un modelo matemático del concepto de prueba elemental en una teoría elemental de tipo $\tau$. A estos objetos matemáticos los llamaremos pruebas formales de tipo $\tau$
        \begin{itemize}
          \item La definición de prueba formal en una teoría de primer orden
        \end{itemize}
  \item Intenta probar matemáticamente que nuestro concepto de prueba formal de tipo $\tau$ es una correcta modelización matemática de la idea intuitiva de prueba elemental en una teoría elemental de tipo $\tau$
        \begin{itemize}
          \item \textit{Teorema de Corrección}: $(\Sigma,\tau)\vdash\varphi$ implica $(\Sigma,\tau)\vDash\varphi$. Es decir,  asegura que nuestro concepto de prueba formal no es demasiado permisivo como para permitir probar sentencias que son falsas en algún modelo de la teoría.
          \item \textit{Teorema de Completitud}: $(\Sigma,\tau)\vDash\varphi$ implica $(\Sigma,\tau)\vdash\varphi$. Es decir, asegura que no puede pasar que se de una prueba elemental de una sentencia $\varphi$ en una teoría $(\Sigma,\tau)$ pero que no haya una prueba formal de $\varphi$ en $(\Sigma,\tau)$.
        \end{itemize}
\end{enumerate}

\end{document}