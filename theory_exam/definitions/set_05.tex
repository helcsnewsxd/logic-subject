\documentclass{article}
\usepackage[utf8]{inputenc}
\usepackage{geometry}
 \geometry{
 a4paper,
 total={170mm,257mm},
 left=20mm,
 top=2mm,
 }
\usepackage{graphicx}
\usepackage{titling}
\usepackage{lipsum}  
\usepackage{lmodern}
\usepackage{amssymb}
\usepackage{amsmath}

\title{Combo 5 de definiciones y convenciones notacionales}
\author{Emanuel Nicolás Herrador}
\date{November 2024}
 
\makeatletter
\def\@maketitle{%
  \newpage
  \null
  \vskip 1em%
  \begin{center}%
  \let \footnote \thanks
    {\LARGE \@title \par}%
    \vskip 1em%
    {\large \@author\quad-\quad \@date}%
  \end{center}%
  \par
  \vskip 1em}
\makeatother

\begin{document}

\maketitle

\section{Notación declaratoria para términos}
\begin{quote}
  Explique la notación declaratoria para términos con sus $3$ convenciones notacionales (convenciones $1,2,5$ de la guía $11$)
\end{quote}
Si $t$ es un término de tipo $\tau$, entonces escribiremos $t=_d t(v_1,\dots,v_n)$ para declarar que $v_1,\dots,v_n$ son variables distintas (con $n\geq 1$) y tales que toda variable que ocurre en $t$ pertenece a $\{v_1,\dots,v_n\}$ (no necesariamente $v_j$ debe ocurrir en $t$).
\newline
El uso de declaraciones de la forma $t=_d t(v_1,\dots,v_n)$ será muy útil cuando se lo combina con ciertas convenciones notacionales que describiremos a continuación:
\begin{itemize}
  \item \textit{Convención 1}: Cuando hayamos hecho la declaración $t=_d t(v_1,\dots,v_n)$, si $P_1,\dots,P_n$ son palabras cualesquiera (no necesariamente términos), entonces $t(P_1,\dots,P_n)$ denotará la palabra que resulta de reemplazar (simultáneamente) cada ocurrencia de $v_1$ en $t$ por $P_1$, cada ocurrencia de $v_2$ en $t$ por $P_2$, etc.
        \newline
        Notar que cuando las palabras $P_i$ son términos, $t(P_1,\dots,P_n)$ es un término.
  \item \textit{Convención 2}: Cuando hayamos declarado $t=_d t(v_1,\dots,v_n)$, si $\mathbf{A}$ es un modelo de tipo $\tau$ y $a_1,\dots,a_n\in A$, entonces con $t^\mathbf{A}[a_1,\dots,a_n]$ denotaremos al elemento $t^\mathbf{A}[\vec{b}]$, donde $\vec{b}$ es una asignación tal que a cada $v_i$ le asigna el valor $a_i$.
  \item \textit{Convención 5}: Cuando hayamos declarado $t=_d t(v_1,\dots,v_n)$ y se de que $t=f(t_1,\dots,t_m)$, con $f\in\mathcal{F}_m$ y $t_1,\dots,t_m\in T^\tau$ únicos, supondremos tácitamente que también hemos hecho las declaraciones $t_1=_d t_1(v_1,\dots,v_n),\dots,t_m=_d t_m(v_1,\dots,v_n)$.
        \newline
        Esto lo podemos hacer ya que obviamente las variables que ocurren en cada uno de los $t_i$ están en $\{v_1,\dots,v_n\}$.
\end{itemize}

\end{document}