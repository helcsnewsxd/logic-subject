\documentclass{article}
\usepackage[utf8]{inputenc}
\usepackage{geometry}
 \geometry{
 a4paper,
 total={170mm,257mm},
 left=20mm,
 top=10mm,
 }
\usepackage{graphicx}
\usepackage{titling}
\usepackage{lipsum}  
\usepackage{lmodern}
\usepackage{amssymb}
\usepackage{amsmath}

\title{Combo 4 de definiciones y convenciones notacionales}
\author{Emanuel Nicolás Herrador}
\date{November 2024}
 
\makeatletter
\def\@maketitle{%
  \newpage
  \null
  \vskip 1em%
  \begin{center}%
  \let \footnote \thanks
    {\LARGE \@title \par}%
    \vskip 1em%
    {\large \@author\quad-\quad \@date}%
  \end{center}%
  \par
  \vskip 1em}
\makeatother

\begin{document}

\maketitle

\section{$(L,s,i,\ ^c,0,1)$ subreticulado complementado de $(L',s',i',\ ^{c'},0',1')$}
\begin{quote}
  Defina "$(L,s,i,\ ^c,0,1)$ es subreticulado complementado de $(L',s',i',\ ^{c'},0',1')$"
\end{quote}
Dados reticulados complementados $(L,s,i,\ ^c,0,1)$ y $(L',s',i',\ ^{c'},0',1')$ diremos que $(L,s,i,\ ^c,0,1)$ es un subreticulado complementado de $(L',s',i',\ ^{c'},0',1')$ si se dan las siguientes condiciones:
\begin{enumerate}
  \item $L\subseteq L'$
  \item $L$ es cerrado bajo las operaciones $s',i',\ ^{c'}$
  \item $0=0'$ y $1=1'$
  \item $s=s'|_{L\times L},\ i=i'|_{L\times L}$ y $^c=\ ^{c'}|_{L}$
\end{enumerate}

\section{$\mathbf{A}\vDash\varphi[\vec{a}]$}
\begin{quote}
  Defina $\mathbf{A}\vDash\varphi[\vec{a}]$ (versión absoluta, no dependiente de una declaración previa, i.e., $\vec{a}\in A^N$. No hace falta definir $t^\mathbf{A}[\vec{a}]$)
\end{quote}
Sea $\mathbf{A}$ una estructura de tipo $\tau$, $\vec{a}$ una asignación y $\varphi\in F^\tau$, definiremos recursivamente la relación $\mathbf{A}\vDash\varphi[\vec{a}]$ (escribiremos $\mathbf{A}\nvDash\varphi[\vec{a}]$ para expresar que no se da $\mathbf{A}\vDash\varphi[\vec{a}]$):
\begin{enumerate}
  \item Si $\varphi=(t\equiv s)$: $\mathbf{A}\vDash\varphi[\vec{a}]$ sii $t^\mathbf{A}[\vec{a}]=s^\mathbf{A}[\vec{a}]$
  \item Si $\varphi=r(t_1,\dots,t_m)$: $\mathbf{A}\vDash\varphi[\vec{a}]$ sii $(t^\mathbf{A}_1[\vec{a}],\dots,t^\mathbf{A}_m[\vec{a}])\in i(r)$
  \item Si $\varphi=(\varphi_1\land\varphi_2)$: $\mathbf{A}\vDash\varphi[\vec{a}]$ sii $\mathbf{A}\vDash\varphi_1[\vec{a}]$ y $\mathbf{A}\vDash\varphi_2[\vec{a}]$
  \item Si $\varphi=(\varphi_1\lor\varphi_2)$: $\mathbf{A}\vDash\varphi[\vec{a}]$ sii $\mathbf{A}\vDash\varphi_1[\vec{a}]$ o $\mathbf{A}\vDash\varphi_2[\vec{a}]$
  \item Si $\varphi=(\varphi_1\to\varphi_2)$: $\mathbf{A}\vDash\varphi[\vec{a}]$ sii $\mathbf{A}\nvDash\varphi_1[\vec{a}]$ o $\mathbf{A}\vDash\varphi_2[\vec{a}]$
  \item Si $\varphi=(\varphi_1\leftrightarrow\varphi_2)$: $\mathbf{A}\vDash\varphi[\vec{a}]$ sii se dan ($\mathbf{A}\vDash\varphi_1[\vec{a}]$ y $\mathbf{A}\vDash\varphi_2[\vec{a}]$) o se dan ($\mathbf{A}\nvDash\varphi_1[\vec{a}]$ y $\mathbf{A}\nvDash\varphi_2[\vec{a}]$)
  \item Si $\varphi=\neg\varphi_1$: $\mathbf{A}\vDash\varphi[\vec{a}]$ sii $\mathbf{A}\nvDash\varphi_1[\vec{a}]$
  \item Si $\varphi=\forall x_i\varphi_1$: $\mathbf{A}\vDash\varphi[\vec{a}]$ sii $\forall a\in A,\ \mathbf{A}\vDash\varphi_1[\downarrow^a_i(\vec{a})]$
  \item Si $\varphi=\exists x_i\varphi_1$: $\mathbf{A}\vDash\varphi[\vec{a}]$ sii $\exists a\in A:\mathbf{A}\vDash\varphi_1[\downarrow^a_i(\vec{a})]$
\end{enumerate}
Cuando se de $\mathbf{A}\vDash\varphi[\vec{a}]$ diremos que la estructura $\mathbf{A}$ satisface $\varphi$ en la asignación $\vec{a}$ y en tal caso diremos que $\varphi$ es verdadera en $\mathbf{A}$ para la asignación $\vec{a}$.
\newline
Cuando no se de $\mathbf{A}\vDash\varphi[\vec{a}]$ diremos que la estructura $\mathbf{A}$ no satisface $\varphi$ en la asignación $\vec{a}$ y en tal caso diremos que $\varphi$ es falsa en $\mathbf{A}$ para la asignación $\vec{a}$.

\section{$v$ ocurre libremente en $\varphi$ a partir de $i$}
\begin{quote}
  Defina la relación "$v$ ocurre libremente en $\varphi$ a partir de $i$"
\end{quote}
Dadas palabras $\alpha,\beta\in\Sigma^*$, con $|\alpha|,|\beta|\geq 1$ y un natural $i\in\{1,\dots,|\beta|\}$, se dice que $\alpha$ ocurre a partir de $i$ en $\beta$ cuando se de que existan palabras $\delta,\gamma$ tales que $\beta=\delta\alpha\gamma$ y $|\delta|=i-1$.
\newline
Definamos recursivamente la relación "$v$ ocurre libremente en $\varphi$ a partir de $i$", donde $v\in Var,\varphi\in F^\tau$ e $i\in\{1,\dots,|\varphi|\}$, de la siguiente manera:
\begin{enumerate}
  \item Si $\varphi$ es atómica (i.e., de la forma $(t\equiv s)$ o $r(t_1,\dots,t_n)$), entonces $v$ ocurre libremente en $\varphi$ a partir de $i$ sii $v$ ocurre en $\varphi$ a partir de $i$
  \item Si $\varphi=(\varphi_1\eta\varphi_2)$ con $\eta\in\{\land,\lor,\to,\leftrightarrow\}$, entonces $v$ ocurre libremente en $\varphi$ a partir de $i$ sii se da alguna de las siguientes:
        \begin{enumerate}
          \item $v$ ocurre libremente en $\varphi_1$ a partir de $i-1$
          \item $v$ ocurre libremente en $\varphi_2$ a partir de $i-|(\varphi_1\eta|$
        \end{enumerate}
  \item Si $\varphi=\neg\varphi_1$, entonces $v$ ocurre libremente en $\varphi$ a partir de $i$ sii $v$ ocurre libremente en $\varphi_1$ a partir de $i-1$
  \item Si $\varphi=Qw\varphi_1$, con $Q\in\{\forall,\exists\}$, entonces $v$ ocurre libremente en $\varphi$ a partir de $i$ sii $v\neq w$ y $v$ ocurre libremente en $\varphi_1$ a partir de $i-|Qw|$
\end{enumerate}

\section{Reticulado cuaterna}
\begin{quote}
  Defina reticulado cuaterna
\end{quote}
Por un reticulado cuaterna entenderemos una $4$-upla $(L,s,i,\leq)$ tal que $L$ es un conjunto no vacío, $s$ e $i$ son operaciones binarias sobre $L$, $\leq$ es una relación binaria sobre $L$ y se cumplen las siguientes propiedades:
\begin{equation*}
  \begin{alignedat}{2}
    1. & \quad \forall x\in L,     &  & \quad x\leq x                                       \\
    2. & \quad \forall x,y,z\in L, &  & \quad x\leq y\land y\leq z\Rightarrow x\leq z       \\
    3. & \quad \forall x,y\in L,   &  & \quad x\leq y\land y\leq x\Rightarrow x=y           \\
    4. & \quad \forall x,y\in L,   &  & \quad x\leq x\ s\ y\land y\leq x\ s\ y              \\
    5. & \quad \forall x,y,z\in L, &  & \quad x\leq z\land y\leq z\Rightarrow x\ s\ y\leq z \\
    6. & \quad \forall x,y\in L,   &  & \quad x\ i\ y\leq x\land x\ i\ y\leq y              \\
    7. & \quad \forall x,y,z\in L, &  & \quad z\leq x\land z\leq y\Rightarrow z\leq x\ i\ y
  \end{alignedat}
\end{equation*}

\end{document}